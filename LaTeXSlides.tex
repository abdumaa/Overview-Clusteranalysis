\documentclass{beamer}

\usepackage[utf8]{inputenc}
\usepackage[german]{babel}
\usepackage{lmodern}
\usepackage{amsmath}
\usepackage{amssymb}
\usepackage{amstext}
\usepackage{amsfonts}
\usepackage{mathrsfs}


\usetheme{CambridgeUS}

\title{Hierarchische Clusteranalyse}
\subtitle{Probleme der hierarchischen Clusteranalyse bei qualitativen Merkmalen}
\author[A. Maarouf]{Abdurahman Maarouf\inst{1}}
\institute[Uni Bonn]{
 \inst{1}Universität Bonn}

\AtBeginSection[]{
 \frame{
   \frametitle{Inhaltsverzeichnis}
   \tableofcontents[currentsection]
 }
} 

  
 
\begin{document}
\frame{\titlepage}

\frame{
 \frametitle{Inhaltsverzeichnis}
 \tableofcontents
}





\section{Einführung}

\subsection{Problemstellung}

\begin{frame}
 \centerline{Problemstellung}
\end{frame}

\begin{frame}
 \frametitle{Ähnlichkeiten erkennen}

 \begin{itemize}
  \item Untersuchung von Ähnlichkeiten unter den \textcolor{blue}{Untersuchungsobjekten}
  \begin {description}
   \item [ Untersuchungsobjekte: ] Personen, Unternehmen, Produkte, etc.
  \end{description}
  \item Sinnvolle Gruppierung anhand der \textcolor{blue}{Merkmalsvariablen}
  \begin {description}
   \item [ Merkmalsvariablen: ] Geschlecht, Alter, Größe, Wohnort, etc.
  \end{description}
  \item Anwendung in vielen Bereichen (z.B. Medizin, Soziologie, Biologie, Wirtschaftswissenschaften, Alltag)
 \end{itemize}
\end{frame}

\subsection{Ziel und Funktion einer Clusteranalyse}

\begin{frame}
 \centerline{Ziel und Funktion einer Clusteranalyse}
\end{frame}

\begin{frame}
 \frametitle{Clusteranalyse als exploratives Verfahren der multivariaten Datenanalyse}
 \begin{itemize}

  \item Gruppen sind im Ausgangspunkt \textcolor{blue}{unbekannt}

  \item Gruppen werden erst durch das Clusterverfahren herbeigeführt

  \item Hierbei werden \textcolor{blue}{alle} Merkmalsvariablen der Untersuchungsobjekte \textcolor{blue}{gleichzeitig} berücksichtigt

  \item Strukturierung und Gruppierung der Untersuchungsobjekte im Hinblick auf folgende Kriterien:

  \begin{block}{Kriterium 1: Homogenität innerhalb der Klassen}
   Objekte in einer Klasse sollen möglichst ähnlich sein
  \end{block}
 
  \begin{block}{Kriterium 2: Heterogenität zwischen den Klassen}
   Verschiedene Klassen sollen möglichst unterschiedliche Objekte enthalten
  \end{block}

 \end{itemize}
\end{frame}

\subsection{Aufbau einer Clusteranalyse}

\begin{frame}
 \centerline{Aufbau einer Clusteranalyse}
\end{frame}

\begin{frame}
 \frametitle{Partitionierendes vs. Hierarchisches Clustering}
 \textcolor{blue}{Partitionierendes Clustering:} \\
 \begin{itemize}
  \item Annahme einer gegebenen Gruppierung und einer \textcolor{blue}{festen Anzahl an Gruppen}

  \item Objekte werden den Gruppen so zugeordnet, dass eine gegebene \textcolor{blue}{Zielfunktion ihr Optimum} erreicht
 \end{itemize}

\textcolor{blue}{Hierarchisches Clustering:} \\
 \begin{itemize}
  \item \textcolor{red}{Agglomerative Algorithmen:} Zusammenfassung der Objekte in Gruppen %Fokus der Präsentation liegt auf das agglomerative Verfahren der hierarchischen Clusteranalyse

  \item \textcolor{red}{Divisive Algorithmen:} Aufteilung der Gesamtheit in Gruppen
 \end{itemize}

\end{frame}

\begin{frame}
 \frametitle{Aufbau des hierarchischen Clustering}

 \begin{block}{1. Bestimmung der Ähnlichkeit}
  2 Objekte werden auf Ähnlichkeiten/Unterschiede untersucht. \
  Ähnlichkeiten/Unterschiede werden mit einem \textcolor{blue}{Proximitätsmaß} gemessen.  
 \end{block}

 \begin{block}{2. Auswahl des Fusionierungsalgorithmus}
  Objekte werden anhand ihrer Proximitätsmaße zu Gruppen zusammengefasst (\textcolor{blue}{agglomerativ}).   \\
  Dabei werden Kriterien 1 und 2 so weit wie möglich erfüllt.
 \end{block}

 \begin{block}{3. Bestimmung der Clusteranzahl}
  Welche Anzahl an Clustern ist die \glqq beste\grqq Lösung? \\
  \textcolor{blue}{Handhabbarkeit} vs \textcolor{blue}{Homogenitätsanforderung} 
 \end{block}

\end{frame}






\section{Ähnlichkeits- und Distanzfunktionen}

\subsection{Definition}

\begin{frame}
 \centerline{Definition}
\end{frame}

\begin{frame}
 \frametitle{Ähnlichkeits- und Distanzmaße}

 \textcolor{blue}{Ähnlichkeitsmaße:} \\
 \begin{itemize}
  \item Je größer der Wert desto ähnlicher sind die zwei Objekte
 \end{itemize}

\textcolor{blue}{Distanzsmaße:} \\
 \begin{itemize}
  \item Je größer der Wert desto unähnlicher sind die zwei Objekte
  \item Falls Objekte vollkommen identisch sind liegt der Wert bei 0
 \end{itemize}

 Wahl des Proximitätsmaßes hängt von dem \textcolor{blue}{Skalenniveau} des Merkmals ab. \\
 Hier werden Variablen mit binären (Ähnlichkeitsmaß) und nominalen (Distanzmaß) Skalenniveaus betrachtet. \\



\end{frame}



\subsection{Ähnlichkeitsfunktionen bei binären Merkmalen}

\begin{frame}
 \centerline{Ähnlichkeitsfunktionen bei binären Merkmalen}
\end{frame}

\begin{frame}
 \frametitle{Von Rohdatenmatrizen zu Ähnlichkeitsmatrizen}
 \textcolor{blue}{Rohdatenmatrix:}
 \bordermatrix{
  & Merkmal_1	& Merkmal_2   & Merkmal_3   \cr
  Objekt_1 & x_{11} & x_{12} & x_{13} & \cr
  Objekt_2 & x_{21} & x_{22} & x_{23} & \cr
  Objekt_3 & x_{31} & x_{32} & x_{33} & \cr
  Objekt_4 & x_{41} & x_{42} & x_{43} & \cr
 } \\
 \ \\
 \centerline{\textcolor{red}{wird transformiert zu:}}
 \ \\
 \textcolor{blue}{Ähnlichkeitsmatrix:}
 \bordermatrix{
  & Objekt_1	& Objekt_2   & Objekt_3 & Objekt_4   \cr
  Objekt_1 & S_{11} & S_{12} & S_{13} & S_{14} \cr
  Objekt_2 & S_{21} & S_{22} & S_{23} & S_{24} \cr
  Objekt_3 & S_{31} & S_{32} & S_{33} & S_{34} \cr
  Objekt_4 & S_{41} & S_{42} & S_{43} & S_{44} \cr
 } \\
 \ \\
 wobei S\textsubscript{ij} den Ähnlichkeitswert der Objekte i und j darstellt.
 

\end{frame}

\begin{frame}
 \frametitle{Berechnung des Ähnlichkeitswerts zweier Objekte}
 \textcolor{blue}{Allgemeines Ähnlichkeitsmaß:}
 \begin{equation}
  \ S_{ij}=\frac{a+\delta d}{a+\delta d+\gamma (b+c)}
 \end{equation}
 wobei: \\
 \begin{itemize}
  \item S\textsubscript{ij}: Ähnlichkeitswert der Objekte i und j
  \item $\delta /\gamma$ : mögliche Gewichtungsfaktoren (abhängig vom \textcolor{blue}{Ähnlichkeitskoeffizienten})
  \item a: Anzahl der Merkmale, die \textcolor{blue}{bei beiden} Objekten vorhanden sind
  \item b: Anzahl der Merkmale, die \textcolor{blue}{nur} bei Objekt j vorhanden sind
  \item c: Anzahl der Merkmale, die \textcolor{blue}{nur} bei Objekt i vorhanden sind
  \item d: Anzahl der Merkmale, die \textcolor{blue}{bei beiden} Objekten \textcolor{blue}{nicht} vorhanden sind
 \end{itemize}
 gilt.
\end{frame}

\begin{frame}
 \frametitle{Ähnlichkeitskoeffizienten bei binären Merkmalen}
 \begin{enumerate}	
	\item \textcolor{blue}{Jaccard-Koeffizient:} \textcolor{red}{$\delta =0$ ; $\gamma =1$}
 				\begin{equation}
  				 \ S_{ij}=\frac{a}{a+b+c}
 				\end{equation}
	\item \textcolor{blue}{M-Koeffizient:} \textcolor{red}{$\delta =1$ ; $\gamma =1$}
 				\begin{equation}
  				 \ S_{ij}=\frac{a+d}{M} ;     M=a+b+c+d
 				\end{equation}
	\item \textcolor{blue}{Russel und Rao-Koeffizient:}
 				\begin{equation}
  				 \ S_{ij}=\frac{a}{M} ;     M=a+b+c+d
 				\end{equation}
 \end{enumerate}

\end{frame}

\begin{frame}
 \frametitle{Auswahl des Ähnlichkeitskoeffizienten}

 \begin{itemize}
  \item \textcolor{blue}{Kein} Ähnlichkeitskoeffizient allgemeingültig vorziehbar
  \item Alle drei Ähnlichkeitskoeffizienten \textcolor{blue}{gleich}, falls \textcolor{red}{$d=0$}
  \item Falls \textcolor{red}{$d>0$} gilt, ist der \textcolor{blue}{M-Koeffizient am größten} und der \textcolor{blue}{RR-Koeffizient am kleinsten}
  \item Bedeutung von \glqq Merkmal nicht vorhanden\grqq bestimmt die Auswahl des Ähnlichkeitskoeffizienten
  \item \textcolor{blue}{M-Koeffizient:} Wenn \glqq Merkmal nicht vorhanden\grqq die \textcolor{blue}{gleiche Aussagekraft} hat wie \glqq Merkmal vorhanden\grqq
  \item \textcolor{blue}{Jaccard-Koeffizient:} Wenn \glqq Merkmal nicht vorhanden\grqq \textcolor{blue}{nicht} die gleiche Aussagekraft hat (Bsp: Deutsch vs. Nicht-Deutsch)

 \end{itemize}

\end{frame}

\subsection{Ähnlichkeits-/ Distanzfunktionen bei nominalen Merkmalen}

\begin{frame}
 \centerline{Ähnlichkeits-/ Distanzfunktionen bei nominalen Merkmalen}
\end{frame}

\begin{frame}
 \frametitle{Variante 1: Transformation in binäre Variable}
 \begin{itemize}
  \item Nominale Merkmale können in \textcolor{blue}{binäre Hilfsvariablen} zerlegt werden
  \item (Deutsch, Französisch, Spanisch,...) $\Rightarrow$ (Deutsch, Nicht-Deutsch)
  \item Restliche Vorgehensweise \textcolor{blue}{gleich} wie bei binären Variablen
  \item \textcolor{red}{Achtung:} Je höher die Anzahl an Merkmalsausprägungen desto \textcolor{blue}{verzerrter} die binären Proximitätsmaße! \\
  $\Rightarrow$ Genauere Ergebnisse mit \textcolor{blue}{Variante 2: Analyse von Häufigkeiten}

 \end{itemize}
\end{frame}

\begin{frame}
 \frametitle{Variante 2: Analyse von Häufigkeiten}
 \textcolor{blue}{Rohdatenmatrix:}
 \bordermatrix{
  & Merkmal_1	& Merkmal_2   & Merkmal_3   \cr
  Objekt_1 & x_{11} & x_{12} & x_{13} & \cr
  Objekt_2 & x_{21} & x_{22} & x_{23} & \cr
  Objekt_3 & x_{31} & x_{32} & x_{33} & \cr
  Objekt_4 & x_{41} & x_{42} & x_{43} & \cr
 } \\
 \ \\
 \centerline{\textcolor{red}{wird transformiert zu:}}
 \ \\
 \textcolor{blue}{Distanzmatrix:}
 \bordermatrix{
  & Objekt_1	& Objekt_2   & Objekt_3 & Objekt_4   \cr
  Objekt_1 & D_{11} & D_{12} & D_{13} & D_{14} \cr
  Objekt_2 & D_{21} & D_{22} & D_{23} & D_{24} \cr
  Objekt_3 & D_{31} & D_{32} & D_{33} & D_{34} \cr
  Objekt_4 & D_{41} & D_{42} & D_{43} & D_{44} \cr
 } \\
 \ \\
 wobei D\textsubscript{ij} den Distanzwert der Objekte i und j darstellt. \\
 Distanzwerte werden mithilfe des \textcolor{blue}{$\chi^2$-Maß} berechnet.
 
\end{frame}







\section{Clusteranalysealgorithmen}

\subsection{Auswahl des Fusionierungsalgorithmus}

\begin{frame}
 \centerline{Auswahl des Fusionierungsalgorithmus}
\end{frame}

\begin{frame}
 \frametitle{Agglomerative hierarchische Algorithmen}
 \begin{itemize}
  \item Gewonnene Distanz- bzw. Ähnlichkeitsmatrix bilden den \textcolor{blue}{Ausgangspunkt} der Clusteralgorithmen
  \item \textcolor{blue}{Breites Spektrum} an Clusteralgorithmen
  \item Schwerpunkt auf \textcolor{blue}{agglomerative hierarchische Algorithmen}, da sie in der Praxis häufig zur Anwendung kommen
  \item Cluster werden anhand der Ähnlichkeits- bzw. Distanzwerte erstellt.
  \item Im Ausgangspunkt stellt \textcolor{blue}{jedes Objekt ein Cluster} dar 

 \end{itemize}
\end{frame}

\subsection{Hierarchische Verfahren}

\begin{frame}
 \centerline{Hierarchische Verfahren}
\end{frame}

\begin{frame}
 \frametitle{Ablauf der agglomerativen Verfahren}

 \begin{block}{Schritt 1}
  Ausgangssituation: Jedes Objekt stellt ein Cluster da
 \end{block}

 \begin{block}{Schritt 2}
  Für alle Objekte werden paarweise Ähnlichkeits- bzw. Distanzwerte bestimmt
 \end{block}

 \begin{block}{Schritt 3}
  Die beiden Cluster mit dem größten Ähnlichkeitswert bzw. kleinsten Distanzwert werden gesucht
 \end{block}


\end{frame}

\begin{frame}
 \frametitle{Ablauf der agglomerativen Verfahren}
 
 \begin{block}{Schritt 4}
  Die beiden Cluster  mit dem größten Ähnlichkeitswert bzw. kleinsten Distanzwert werden zu einem neuen Cluster zusammengefasst
 \end{block}

 \begin{block}{Schritt 5}
  Neue Ähnlichkeits- bzw. Distanzwerte werden mit den übrigen Gruppen berechnet \\
  $\Rightarrow$ \textcolor{blue}{reduzierte Ähnlichkeits- bzw. Distanzmatrix}
 \end{block}

 \begin{block}{Schritt 6}
  \textcolor{blue}{Schritt 3} bis \textcolor{blue}{Schritt 5} werden solange wiederholt, bis es nur ein Cluster gibt \\
  $\Rightarrow$ \textcolor{blue}{Ein-Cluster-Lösung}
 \end{block}

\end{frame}

\begin{frame}
 \frametitle{Zu Schritt 5: Berechnung der neuen Distanzwerte}

 \begin{block}{Schritt 5}
  Neue Ähnlichkeits- bzw. Distanzwerte werden mit den übrigen Gruppen berechnet \\
  $\Rightarrow$ \textcolor{blue}{reduzierte Ähnlichkeits- bzw. Distanzmatrix}
 \end{block}

 \begin{itemize} 
  \item Zusammenfassung von \textcolor{blue}{Cluster X} und \textcolor{blue}{Cluster Y} zu \textcolor{blue}{Cluster X+Y}
  \item Neuer Distanzwert von \textcolor{blue}{Cluster X+Y} und \textcolor{blue}{Cluster Z}:
 \end{itemize}

 \begin{equation}
  \ D(Z;X+Y)=A\cdot D(Z;X)+B\cdot D(Z;Y)+E\cdot D(X;Y)+G\cdot |D(Z;X)-D(Z;Y)|
 \end{equation}
 

\end{frame}

\begin{frame}
 \frametitle{Zu Schritt 5: Berechnung der neuen Distanzwerte}

 \begin{equation*}
  \ D(Z;X+Y)=A\cdot D(Z;X)+B\cdot D(Z;Y)+E\cdot D(X;Y)+G\cdot |D(Z;X)-D(Z;Y)|
 \end{equation*}

 mit

 \begin{equation*}
  \ D(I;J)=\text{Distanz zwischen den Clustern I und J}
 \end{equation*}

 \begin{itemize}
  \item A,B,E und G sind Konstanten
  \item Sie werden vom \textcolor{blue}{agglomerativen Verfahrensalgorithmus} bestimmt
 \end{itemize}

\end{frame}

\begin{frame}
 \frametitle{Agglomerative Verfahrensalgorithmen}

 \begin{enumerate}
  \item \textcolor{blue}{Single-Linkage-Verfahren:} \textcolor{red}{$A=0.5$ ; $B=0.5$ ; $E=0$ ; $G=-0.5$}   %nächstgelegener Nachbar -> min {D(Z;X);D(Z;Y)}

  \begin{equation}
   \ D(Z;X+Y)=0.5\cdot (D(Z;X)+D(Z;Y)-|D(Z;X)-D(Z;Y)|)
  \end{equation}

  \item \textcolor{blue}{Complete-Linkage-Verfahren:} \textcolor{red}{$A=0.5$ ; $B=0.5$ ; $E=0$ ; $G=0.5$}  %entferntester Nachbar -> max  {D(Z;X);D(Z;Y)}

  \begin{equation}
   \ D(Z;X+Y)=0.5\cdot (D(Z;X)+D(Z;Y)+|D(Z;X)-D(Z;Y)|)
  \end{equation}

  \item \textcolor{blue}{Average-Linkage-Verfahren:} \textcolor{red}{$A=0.5$ ; $B=0.5$ ; $E=0$ ; $G=0$} 

  \begin{equation}
   \ D(Z;X+Y)=0.5\cdot (D(Z;X)+D(Z;Y))
  \end{equation}

 \end{enumerate}


\end{frame}

\begin{frame}
 \frametitle{Dendrogramm}

 \centerline{{\includegraphics[height=6cm]{hl_dendrogram.png}}}
 \centerline{Abbildung 1}

\end{frame}








\section{Analyse und Interpretation}

\subsection{Spezielle Probleme}

\begin{frame}
 \centerline{Spezielle Probleme}
\end{frame}

\begin{frame}
 \frametitle{Probleme der praktischen Durchführung}

 \begin{enumerate}
  \item \textcolor{blue}{Große Anzahl an Untersuchungsobjekten} erschweren die Durchführung einer hierarchischen Clusteranalyse
  \item \textcolor{blue}{Große Anzahl an Merkmalsvariablen} erschwert die Interpretation der Ergebnisse
  \item Bei Datenauswertungen \textcolor{blue}{fehlen} häufig Daten oder sind \textcolor{blue}{ungültig}
 \end{enumerate}

\end{frame}

\begin{frame}
 \frametitle{Handhabbarkeit vs. Homogenitätsanforderung}

 \begin{itemize}
  \item Nach dem agglomerativen Verfahren muss die\textcolor{blue}{\glqq beste\grqq Anzahl von Clustern} bestimmt werden
  \item Einerseits Erfüllung der \textcolor{blue}{Homogenitätsanforderung}
  \item Andererseits Maximierung der \textcolor{blue}{Handhabbarkeit}
  \item Beide Anforderungen \textcolor{blue}{gleichzeitig} zu erfüllen fällt schwer \\
  $\Rightarrow$ \textcolor{blue}{Elbow-Kriterium}
 \end{itemize}

\end{frame}

\subsection{Bestimmung der Clusteranzahl}

\begin{frame}
 \centerline{Bestimmung der Clusteranzahl}
\end{frame}

\begin{frame}
 \frametitle{Elbow-Kriterium}
   
 Optische Identifikation eines \glqq Sprungs" (\textcolor{blue}{Elbow}) im Dendrogramm oder im \textcolor{blue}{Scree-Plot}:
 \centerline{{\includegraphics[height=6cm]{scree.png}}}
 \centerline{Abbildung 2}
\end{frame}

\section*{}

\begin{frame}
 \frametitle{Quellen}
 \begin{enumerate}
  \item Detlef Steinhausen:\glqq Clusteranalyse, Einführung in Methoden und Verfahren der automatischen Klassifikation", 1977
  \item Joachim Hartung:\glqq Multivariate Statistik: Lehr- und Handbuch der angewandten Statistik", 1995
  \item Klaus Backhaus:\glqq Multivariate Analysemethoden: eine anwendungsorientierte Einführung", 2016
  \item Chris Fraley:\glqq How Many Clusters? Which Clustering Method? Answers Via Model-Based Cluster Analysis", 1998
  \item Michael Wiedenbeck: \glqq Klassifikation mit Clusteranalyse : grundlegende Techniken hierarchischer und K-means-Verfahren", 2001
 \end{enumerate}
 Abbildungen:
 \begin{enumerate}
  \item Abbildung 1: $http://www.statistics4u.info/fundstat_eng/img/hl_dendrogram.png$
  \item Abbildung 2: $\text{https://kamihoeferl.wordpress.com/2014/05/01/clusteranalyse-mit-r-echts/}$
 \end{enumerate}
\end{frame}

\begin{frame}
 \centerline{Vielen Dank für Ihre Aufmerksamkeit!}
\end{frame}


\end{document}

